\documentclass[16pt]{article}
\usepackage[margin=1.2in]{geometry}
\usepackage{comment}
\usepackage{mathtools}
\usepackage{parskip}
\usepackage{setspace}

\def\sp#1{\textsuperscript{#1}}
\def\sb#1{\textsubscript{#1}}
\parindent 0
\onehalfspacing


\title{COMP3520 Host Dispatcher Design Document}
\author{Justin Ting, jtin2945, 430203826}
\date{May 2015}

\begin{document}

\maketitle

\pagebreak
\section{Memory Allocation Algorithm}

\iffalse\documentclass[12pt]{article}
\usepackage[margin=0.5in]{geometry}
\usepackage{comment}
\usepackage{mathtools}
\usepackage{parskip}

\def\sp#1{\textsuperscript{#1}}
\def\sb#1{\textsubscript{#1}}
\parindent 0

\title{COMP3520 Host Dispatcher Design Document}
\author{Justin Ting, jtin2945, 430203826}
\date{May 2015}

\begin{document}

\maketitle

\pagebreak
\section{Memory Allocation Algorithm}

\iffalse
(Provides a thorough description of ALL prescribed memory allocation schemes - 6 marks)
(Provides an informed discussion of the advantages and disadvantages of these schemes)
For each: pros, cons, examples
a. Describe and discuss what memory allocation algorithms you could have used
and justify your final design choice
\fi

A number of memory allocation algorithms were available to choose from for this assignment, and each one will be detailed below - with a description of what they do, their advantages and disadvantages, and a worked example that will be consistent across each algorithm, using the specific version of the algorithm detailed in each subsection.\\

The algorithms that were allowed to be chosen from were First Fit, Worst Fit, Next Fit, Best Fit, and the Buddy System, all of which at least partially fall under the category of `Dynamic Partioning` (the latter being in a sense a `combination` of fixed and dynamic partioning), in contrast to `Fixed Partioning`. The former allows a variable number of partitions as well as their sizes, while the latter has both a fixed number of partitions as well as their sizes (although fixed sizes does not necessarily indicate fixed \textit{equal} sizes - unequal fixed sizes also still fit the category as well).\\

At this point, part of the narrowing down of algorithm choice has already been done for us due to the limitations of the specifications of the task - the exclusion of Fixed Partioning algorithms means that we automatically avoid some of their main disadvantages, including having a fixed (and hence) inflexible number of processes that can be running at any one time, inefficient use of partition space by smaller jobs, and internal fragmentation within partitions. Not only this, but virtually no real world systems still employ the use of fixed partioning, a clear indicator of its ineffiencies and pitfalls.\footenote{William Stallings `Operating Systems Internals and Design Principles Seventh Edition` 2012. p. 341}\\

For our available class of algorithms, they share a number of properties which for the sake of brevity will be outlined in the introduction here rather than repeated in each subsection. Memory blocks managed through dynamic algorithms do not suffer from \textit{internal fragmentation}, but rather, \textit{external fragmentation}. Whereas internal fragmentation is a more inherent problem that cannot be solved without major modifications to an algorithm itself, external fragmentation is an issue that can be solved more simply (although not without a cost) - through the use of compaction. While internal fragmentation alludes to unused space within fragments, external fragmentation is the space between allocated partitions which are unused (and potentially can't have any pending jobs slot into it). Compaction solves this by periodically (based on a factor determined by the algorithm) by packing all the memory together back into a contiguous block, external fragmentation is at that moment in time eliminated - though this has issues of its own, as it requires both time and processing power, but more importantly, the now-modified locations of memory blocks can't invalidate references that previously existed which programs relied upon.\\

The following algorithms are named based on the way in which they allocate memory to incoming programs' requests.

\subsection{Best Fit}
The \textbf{Best Fit} memory allocation algorithm, as the name suggests, will find the smallest block available that is able to fit the request which it is given, starting from the beginning. This means that each search for an appropriate block is of $O(n)$ complexity, as an entire memory block must be traversed to determine the best-fitting block. This algorithm performs dismally, as always finding the best block means that the resulting \textit{size} of unused space (external fragmentation) is minimised - the wording of which can lead one to incorrectly believe is a positive thing, but in fact, rapidly creates a large number of miniscule, unusable blocks of memory, requiring compaction, which, as described above, is an expensive process which we want to minimise - while the Best Fit algorithm essentially maximises it instead. As a result, the best fit algorithm is in fact the worst performer of the pure dynamic allocation algorithms.\\

\textbf{However}, it is worth noting that despite its theoretical worst performance, this worst-case behaviour is often not observed, and when these cases do not surface simply due to the nature of the programs making requests to a block of memory, it doesn't necessarily perform worse than some of the other dynamic memory allocation algorithms. While some sources will find this point contentious, some (and possibly a majority) will actually claim that research has shown that Best Fit and First Fit aren't actually superior to each other in terms of wasted memory (though the latter holds) 

\subsection{Next Fit}
The \textbf{Next Fit} memory allocation algorithm will remember the last spot in which memory was last allocated (naturally, this will be the beginning if any memory is yet to be allocated), and serve requests by finding the next available block that is sufficiently large - taking $O(n)$ time. Under normal operation, this means that what usually happens is the largest block at the `end` of memory will be quickly fragmented as processes are finishing and their allocated memory freed, and new processes with different requirements take their place. This points to a relatively frequent need to compact memory, though less so than the theoretical worst case that the Best Fit algorithm requires.

\subsection{First Fit}
The \textbf{First Fit} memory allocation algorithm search from the beginning of a memory block to find the first block that is sufficiently large to satisfy the request being made. This traversal has a complexity of $O(n)$. Due to searching from the start every time, it takes longer than Next Fit to require breaking up blocks in the later areas of memory, which also means that theoretically, fragmentation will occur slower and hence compaction doesn't have to happen as frequently, reducing the overhead of the algorithm in this regard. However, though it holds the same complexity as Next Fit for each search, the \textit{actual} time to search is slightly higher as it has to check from the start each time, rather than simply the remaining memory left from the last allocated location. This means that during operation when not in the process of compacting, the First Fit algorithm is slower than the Next Fit algorithm. 

\subsection{Worst Fit}
The \textbf{Worst Fit} memory allocation algorithm will search from the beginning of a full memory block and find the block that is largest in difference to the request actually being made - hence it being named `worst fit`, and takes $O(n)$ time per search. Due to the requirement from searching the entire memory space each time a request is made, its speed during operation is the same speed as that of Best Fit.\footnote{D. Samanta. `Classic Data Structures` 2004. p. 76} However, its major difference to the Best Fit algorithm is that rather than trying to minimise `wasted` space, i.e. maximise small, fragmented blocks, it does the reverse - in place of the theoretical `small, unusable` blocks created by Best Fit, Worst Fit will instead have the largest block remaining possible, leaving them available for later processes that make larger requests for available memory.

\subsection{Buddy System}
The \textbf{Buddy System} memory allocation algorithm differs from the above in that it neither purely uses fixed or dynamic partioning schemes, but rather, a sort of combination of both. This is due to its aim to circumvent the inherent negative aspects of both - to avoid limitation of the number of processes and potential inefficient use of memory of fixed schemes, and the constant need for compaction and its computational cost of dynamic allocation schemes.\\

The buddy system assumes that the total size of available memory is $2^{U}$, where U is a positive integer. To obtain a block of memory to serve a program's request, the smallest block that is greater than or equal to the request size that is also $2^{k}$ where k is a positive integer is allocated. This block is retrieved by taking the full $2^{U}$ block, and spawning two children of size $2^{U-1}$, each of which spawn children of $2^{U-2}$, etc...until the correctly size block is found, at which point it is given to the program requesting it. It is apparent that this represents a binary tree - which is one of the data structures in which the Buddy System can be represented by. Hence, the complexity of allocation and freeing of memory is $log(n)$, a considerable increase in speed over the previous dynamic allocation methods detailed.\\

In its unadulterated form, the reader would be correct in observing one critical flaw with the buddy system - that it further exaggerates the problem of internal fragmentation, arguably beyond that of any reasonably designed fixed partitioning scheme - if we were to start with a 4096MB block of memory, and a program requests 2049MB, that would render the entire block in use, leaving 2047MB empty, yet unusable. This can be remedied through the use of slab allocation - a concept beyond the scope of the course.\\

\subsection{Algorithm Selected}
The algorithm that was selected for implementation in the assignment was the \textbf{Buddy System}. The reasons for this were both its superior speed when allocating, in comparison to the previous dynamic partitioning schemes, and eliminating the issues of external fragmentation, although it instead has issues of internal fragmentation - which can be fixed by other means. In the context of the assignment, however, this fact can be overlooked (or, more accurately, slab allocation does not need to be implemented) - and it is enough to identify that a solution exists for it. The other major advantage of choosing Buddy is that it is computationally cheaper to free memory - and seeing as the assignment specification says up to 1000 jobs can be in a particular instance, the constant freeing of jobs would impact performance (although not in a way that the user would notice given the nature of the application and the way jobs are `assigned`).\\

Upon further investigation, it can also be seen that many real-world systems even today use the Buddy System at least in part, in their memory allocation schemes. For example, the open source library \textit{jemalloc} uses the buddy system\footnote{people.freebsd.org/~jasone/jemalloc/bsdcan2006/jemalloc.pdf}, and is used in very well known projects today, such as the Firefox browser for managing fragmentation during its use on Windows platforms\footnote{https://github.com/jemalloc/jemalloc/wiki/History}, as well as the FreeBSD OS.

\iffalse (Justification of choice - identifies memory allocation algorithm used in code and clearly justifies choice in context of assignment - 2 marks) \fi

\pagebreak
\section{Queuing, Dispatching, Memory Allocation, and Resource Allocation Structures}

\iffalse
(Provides thorough description of ALL dispatcher structures that are relevant to host dispatcher implementation - 5 marks + 1 bonus)

b. Describe and discuss the structures used by the dispatcher for queuing,
dispatching and allocating memory and other resources.

structs, linked lists, [trg]
\fi

All the structures that were used in the implementation of the host dispatcher, including both the main program and the modules which it relied on, comprise of \textit{structs}, \textit{singly linked lists} which were used as queues in all instances where they appeared, \textit{arrays}, and \textit{binary trees}.

\subsection{Queueing and dispatching processes}
\textit{C structs} were used for a Process Control Block(\textit{Pcb}) which held the data for processes, containing their process ID, arguments (for the default `process` process that simply prints incremental `ticks` at a rate of 1/sec), arrival time, priority, remainingcputime, (psuedo) memory required in MB, a C struct containing resource requirements, status, a pointer to a C struct containing memory block information, and a pointer to the next \textit{Pcb}.\\

To actually queue and dispatch these processes (represented using Pcb's holding process info), singly lniked lists were used, as we only ever needed to take Pcb's from the front of queues, never in the middle of one or from the end. A Realtime Queue linked list was used to hold real time processes, an array of Feedback Queues were used to hold the feedback queues, with its indices offset by one compared to each job's actual priority (priority 1 jobs were in the feedback queue at index 0 in the array, priority 2 at index 1, priority 3 at index 2).\\

\subsection{Allocating memory}
As mentioned in \textit{Queueing and dispatching processes}, C structs were used to hold information pertaining to memory blocks (Memory Allocation Blocks, or Mabs) - i.e., their offset in main memory, the block's size, amount allocated (though this does not include memory allocated in any children blocks). Because the Buddy System was used for managing memory, a binary tree was the data structure of choice to maintain the available memory - and hence, each Mab also contains a pointer to its pointer Mab, as well as its left and right child Mabs, where they existed. Two binary trees were maintained - one for user jobs, and one for realtime jobs. 

\subsection{Allocating resources}
Once again, as mentioned in \textit{Queueing and dispatching processes}, C structs were used to hold information regarding required resources for a particular Pcb - i.e., the printers, scanners, modems, and cds it needed. While each Pcb contains a resource struct to keep track of the number of resources required, one more Resource struct existed in the main program to keep track of the available resources in the overall system at any one time.

\subsection{Appropriateness}
Due to C being a non-object oriented language, these were the best structures to use given the problem at hand. Queues are explicitly stated in the assignment spec, and hence queues were needed - implemented as singly linked lists. The only way to hold blocks of information together in C is using structs, hence their use for multiple parts of the program. The very nature of the Buddy System points towards it using a binary tree, which was why that was also used - and many sources verify this too, not the least of which includes the Art of Programming Volume 1.\footnote{D. Knuth. `The Art Of Programming: Volume 1 Fundamental Algorithms, Third Edition` 2013. p. 650 (ebook Edition)}\\

Some alternatives do exist to the structures used. On a more detailed level, while the implementation that this Design Document described used nested structs, all the information from each struct could have been instead all compressed into the one struct - in this way, memory would have been allocated marginally more contiguously, but this difference would have been so minor that it would be more advantageous for the sake of code clarity to separate information into appropriate C structs.\\

On a more fundamental level, the problem at hand would have lended itself nicely to the use of classes in C++ as well - something which could have been emulated in a bastardised form of C, by containing the functions relating to the purpose of each struct with the use of function pointers. However, this goes against common usage of C, as well as arguably violating using `straight` C, as the assignment required.

\pagebreak
\section{Program Structure}

\iffalse
(12 marks + 3 bonus)
(Provides comprehensive description of overall program structure, its modules, and major functions)
(Provides detailed and clear justification of program structure in context of assignment)

c. Describe and discuss the overall structure of your program, describing the
various modules and major functions (description of the function interfaces are
expected)
\fi

\subsection{Overall Program Structure}
The overall program structure of the Host Dispatcher utilises three modules - that of the Process Control Block, the Memory Allocation Block, and Resource block modules. This results in the code base in its entirety representing a modular, logical structure, accurately breaking down the problem into its constituents in terms of each individual module. By structuring the code around the inherent structure of the problem itself, it makes it easier to understand at first glance, as well as to modify when parts of the host dispatcher (or program in general) needs to be modified or extended. This may not be as apparent when looking at this assignment in isolation, but for arugment's sake, if there were subsequent assignments where we received other students' code and were required to perform major extensions on it, well-implemented modularity would make it a much less daunting task. As a better (and more real-world relevant) example, perhaps a much larger program needs to incorporate this relatively miniscule project into it - and if it needed to use indiviual elements of it, the programmer could look directly to the appropriate module, rather than digging through a single, large file, which would likely be a monolithic, unorganised mess.\\

The main program contains several key elements:
\begin{itemize}
\item An \textbf{inputqueue}, which takes in all jobs from an input text file
\item A \textbf{userjobqueue}, which holds all user jobs from the inputqueue
\item An \textit{array of pointers} to \textbf{feedback queues}, containing jobs from the \textbf{userjobqueue} that have been supplied the sufficient amount of memory and resources they require
\item A \textbf{realtimebuffer}, which contains all real tmie jobs from the inputqueue
\item A \textbf{realtimequeue}, which contains jobs from the realtimebuffer that have been given enough memory to run
\end{itemize}

And without getting into any code or even too deep into specifics, the general flow of the program is as follows (note that the following list is not intended to act as instructions to follow to be able to implement the problem at hand to completion and perfection - but rather, to get a concise idea of how the problem was approached - hence, it is missing implementation specific details such as reading in files, specific error handling, actual flow logic, etc.). Steps $2$ and $3$ represent the Real Time Dispatcher module, while steps $4$ and $5$ represent the Feedback Dispatcher module.
\begin{enumerate}
\item Place all jobs from the inputqueue into the \textbf{userjobqueue} and \textbf{realtimebuffers} respectively
\item Allocate memory to jobs at the front of the realtimebuffer as long as it is available
\item Run all jobs in the \textbf{realtimequeue} from beginning to end until completion (hence skipping the following steps until such time is reached)
\item Allocate memory and resources to jobs in the \textbf{userjobqueue} as they become available, and enter it in the appropriate feedback queue (array) based on the specified priority
\item Run the process at the front of the highest priority feedback queue for one quanta, before placing it at the end of the next lower priority feedback queue
\end{enumerate}

The following subsections will detail each individual module and its major functions, and by doing so the way each one ties into the functionality of the program as a whole will become apparent. Note that trivial helper functions have been excluded, as they provide no useful insight into how the program functions, even to someone looking to utilise the program as a whole needing to understand roughly how everything operates and weaves together. Unlike the previous paragraph, however, considerable detail will be given with regards to each modules' functions, as they are the foundation of the main program.

\subsection{Process Control Block (pcb) Module}
\textbf{startPcb}\\
This function takes a Pcb pointer and checks the process' status - it will then either start an unstarted process using C's \textit{fork} and \textit{exec} functinos, or continue a paused one with a \textit{SIGCONT} signal- and this is done by checking the process ID (which is stored in a Pcb's struct) - a zero ID indicates a new process, while a non-zero one indicates it already exists but has simply been paused by a \textit{SIGTSTP} signal. The function's return value indicates whether it carried out its operation successfully - a vaild pointer indicates success, while a \textit{NULL} pointer means that the start/restart operation it attempted failed. This function gets called whenever a process is first being executed from a feedback queue, or being resume from a paused state. As such, there is no `resumePcb` function, as the reader may have observed.\\

\textbf{suspendPcb}\\
This function takes a Pcb pointer and sends the process it stores a \textit{SIGTSTP} signal, causing it to pause. The function will similarly return a NULL pointer if the signal fails. This function is called only on jobs in feedback queues, each time it has executed for one quanta.\\

\textbf{terminatePcb}\\
This function takes a Pcb pointer and sends the process it stores a \textit{SIGINT} signal, causing it to die. It is only called when a program's remainingcputime has reached zero.\\

\textbf{enqPcb}\\
This function takes a pointer to the head of a queue (specifically, a Pcb), and a Pcb pointer to the process block that is to be queued. The head of the existing queue will be stored in a temporary variable, and then modified to point at the back of the queue, allowing the last item in the queue to then be pointed at the target Pcb to be enqueued. The \textit{head} of this new queue is then returned, or in the case where a queue is yet to be created, simply the \textit{process} to be enqueued is returned instead - as it is the head of this newly created queue. This function is called whenever Pcbs need to be enqueued - whether it is into the \textbf{userjobqueue}, from this to the \textbf{realtimebuffer}/\textbf{userjobqueue}, to the \textbf{realtimequeue}/\textbf{feedbackqueues} respectively, or from one \textbf{feedbackqueue} to another.\\

\textbf{deqPcb}\\
This function takes a double pointer to the head of a queue (specifically, a Pcb), and takes the frontmost item off the queue by assigning a temporary pointer to the current head, and then pointing the head to the next item (in this case, the next of the temporary Pcb pointer). Hence, the new head after dequeuing will be returned if successful, or NULL returned if unsuccessful, when a queue is already empty. With the exception of when the inputqueue is being loaded with data from an external text file, deqPcb will always be called in pairs with enqPcb - a Pcb is only enqueued onto one queue after being dequeued from another.\\

\textbf{checkRsrcs}\\
This function takes a pointer to a Pcb and checks whether it requests valid resources. The role of this function is twofold, as it checks for both realtime and user jobs. It returns true for the former only if no resources are requested, whereas it returns true for the latter so long as the requested amount doesn't exceed the system's total available resources (not simply the amount available at the time - that allocation is handled in the main program). This function is called whenever a process is trying to obtain resources in order to be queued to run.
\\

\textbf{checkMem}\\
This function takes a pointer to a Pcb and checks whether it requests valid memory. The role of this function is similarly twofold, as it checks for both realtime and user jobs. It returns true for the former when memory requested doesn't exceed 64mb, whereas it returns true for the latter so long as it doesn't return more than half the available size. This seemingly abnormal restriction is because of the naive implementation of the Buddy System, and reserved memory for realtime jobs - the largest available block for a system with 1024MB memory using this implementation of the Buddy System is only 512MB. This function is called whenever processes are trying to reserve a block of memory in order to be queued to run.\\

At this point, multiple justifications are in order regarding some of the design choices made above. 
\begin{itemize}
\item There is a generalised check function despite realtime jobs and user jobs requiring different checks - the programmer using the interface should not have to worry about this, as it involves looking at information in the Pcb (determining its type based on priority), or checking the surroung logic in the code - rather, they should be able to call a check on any Pcb and trust that it will determine validity of its memory/resource requests regardless of type
\item The check functions do not simultaneously allocate memory/resources if they are available. This is necessary, as both resources and memory must be available at any point in time before they are allocated - if each pre-allocates them within their isolated functions, we will end up with cases where one allocates memory/resources and the other doesn't because it wasn't able to - resulting in either a bug, or extra code to do otherwise unnecessary freeing if both conditions are satisfied at once
\item This point must may have occurred to the reader as soon as they happened upon the check functions - why are resource and memory checks not in the resource and memory modules? This is justified and even preferred, because first of all - they should take pointers to Pcbs and not Rsrc/Mab blocks specifically, as the programmer using these interfaces should not have to worry about such specifics. Secondly, given the previous assertion, we can't place the functions in the memory/resource modules as circular dependencies would arise as a result
\end {itemize}

\subsection{Memory Allocation Block (mab) Module}
\textbf{blockSizeNeeded}\\
This function takes an integer, and finds the smallest power of two that is greater than it, and returns the value. This function is used throughout the Memory Allocation Block module to determine sizes needed via the Buddy System.\\

\textbf{memChk}\\
This function takes a pointer to a Mab - the root node of the Buddy tree, and an integer, the size being requested, and checks if an available block of such a size exists. It does this by recursing into the left then the right trees, seeing if there is a valid block that can be allocated. It will continue to spawn child nodes at each node, each containing half the size of its parent, until one is found, in which case the pointer to this valid block is returned. If no valid block can be found in the tree, the function returns NULL. It is called when a process wants to first check if there is available memory before allocating it.\\

\textbf{memAlloc}\\
This function takes a pointer to a Mab and an integer whose size is that needed by a process, and essentially makes a call to memChk and allocates it if a valid block is returned. A pointer to the Mab containing the allocated block is returned if successful, or NULL otherwise. The function is called once resource/memory requirements have been confirmed to be satisfied, and memory can be safely allocated.\\

\textbf{memFree}\\
This function takes a pointer to a Mab - the one which is to be freed. It marks its allocated amount as zero, then checks that it doesn't have children (which would indicate that one of its child or indirect child nodes has allocated memory, due to the way this module is written), and also that its buddy doesn't have children. If this is the case, both the block to be freed and its buddy will be freed - and this process is repeated recursively until all free blocks have been merged back into one larger block. Due to the method calls made from \textbf{memFree}, it is guarded from errors, and the function simply returns a NULL pointer once it has freed all it can. This is called in conjunction with whenever processes are terminated.\\

\textbf{createUserMem}\\
This function takes no arguments, and creates a block of memory equal in size to all the available memory, and returns a pointer to the Mab block - which acts as the root node of the Buddy System memory tree.\\

\textbf{createRTMem}\\
This function takes no arguments, and creates a block of memory equal to the max size available for real time jobs (in our case, 64MB), and returns a pointer to the Mab block - the root node of the Buddy System memory tree.\\

Once again, some justifications need to be made regarding the design choices of this module:
\begin{itemize}
\item It may appear that the Buddy System memory tree is being damaged when checks are performed, which may render later use of it invalid - however, due to how it is implemented here, this is inevitable - if a check is performed but an allocation not made, the changes to the tree must be rolled back. While this may seem expensive, these operations have a complexity of $O(logn)$ - two operations of this complexity are still far lower in complexity than the consistently $O(n)$ operations of previously mentioned dynamic partionining schemes
\item While the previous module combined user and realtime checks in a single function, operations exclusive each have been separated here - this is because they require different properties on creation, and a generic interface would have resulted in the programmer using the interface to explicitly specify the size of the root node - an implementation detail which they should be shielded from
\end{itemize}

\subsection{Resource Block (rsrc) Module}


\pagebreak
\section{Discussion - Use of Multilevel Dispatching Scheme}

The feedback queue component of the task utilises a multi-level dispatching scheme. These are important for process management because they allow the prioritising of different types of processes requiring different resources (e.g. CPU bound or I/O bound) to be adequately intricate such that throughput and performance are maximised. In recent times (though not technologically speaking), Windows NT was one particular system that utilised such a scheme. This ties in with one key weakness of the algorithm being used in the task at hand - equal quanta are assigned to CPU-bound tasks are they are to those requiring I/O operations - but because by their nature, I/O tasks are more expensive and time consuming, it is possible that quite often they only ever use part of their quanta, sacrificing the rest without ever getting it back, which inadvertently favours CPU-bound tasks over I/O-bound ones. One possible solution to this is the one which Windows NT employed - by adjusting priorities (naturally, priorities in a real operating system are likely to be more fine-grained then the coarse 3-level one used in our dispatcher) based on whether a process favoured I/O or CPU intensive tasks, and gave higher priority to the former.\footnote{"A Tale of Two Schedulers Windows NT and Windows CE", http://web.archive.org/web/20120722015555/http://sriramk.com/schedulers.html}\\

This is not the only solution to the problem, however - rather than simply assigning higher priority to I/O bound processes to try and balance the scales in terms of time given to processes, which would involve estimating how long certain I/O operations would take relative to their CPU counterparts before adjusting priorities accordingly which would provide them even time, a rather complex calculation, there exists a simpler alternative. This would be to actually take the time spent by an I/O-bound process after the fact, and then take the leftover time, i.e. subtracting the time spent by the process from the quanta that was actually available to it, and storing that value for that particular process. The I/O-bound process, if it hadn't used up its quanta, is then placed on a special queue for I/O-bound processes which are allowed to execute for a period equal to the sum of time equal to its leftover each time a quanta wasn't fully utilised. This scheme is also utilised in real-world systems - one particular example being Solaris.\footnote{pages.cs.wisc.edu/.../solaris-notes.pdf}.\\

Both Window NT and Solaris present viable solutions to problems which are very much inherent in our implementation of a multilevel dispatcheing scheme - that being that we don't intentionally discriminate between I/O-bound and CPU-bound processes, and as such the natural CPU-bound skew will arise, causing potential starvation for I/O-bound processes. This only accounts for one flaw in our scheme - an even larger, more critical problem, arguably, is the high possibility of starvation in our naive implementation. In our three-level dispatcher queue, once a process has reached priority level three, it stays there with no way to increase its priority - this means that if a constant stream of processes request resources that are available (likely to be low resource-requirement with low total execution times), processes that drop to the lowest priority will never get serviced. While the previous issue addressed starvation of I/O-bound processes specifically, we have now identified that our model also has the issue of starvation of processes in general.\\

The most obvious solution to this issue is the use of aging - which observes the age of a process within its particular level in a multilevel feedback queue, and based on metrics defined by the designer of the system, can raise the priority of a process depending on its age within a particular queue, to prevent starvation. Depending on how old it is, it could be raised by $k$ priorities, proportional to its age. While the current implementation doesn't lend itself perfectly to this solution, the fix to do so is an easy one. Because the \textbf{realtimequeue} is currently separate from the array of feedback queues, and the structure of the main program unquestioningly gives absolute priority to jobs in the \textbf{realtimequeue}, raising priorities of user jobs won't make a difference if there exists a constant stream of realtime jobs. Combining the \textbf{realtimequeue}s and \textbf{feedbackqueue}s into a single array could present a viable remedy - first by having realtime jobs at index $0$in the array, and processes with current priority $k$ being in queues of $1...n$ where $1 <= k <= n$. For queues at index $1$ onwards, an age constant could then be used for each process in the queue where once it hits a certain number, will increase its priority by one and hence enter the queue at priority $k-1$ where $k$ is the queue's index which the process waited in before its age reached the threshold. An example of this in practice would be that if priorities ranged from 0 to 127\footnote{Silbershatz, Galvin, Gagne Operating System Principles, 7th ed., p.163}, we could raise the priority of a waiting process by one every second (the amount suggested in the previously referenced book was 15 minutes, an obtusely unrealistic amount for an average user system) - in which case, even a priority 127 task would be executed in a little over two minutes.

\iffalse
(20 marks + 5 bonus)
(Demonstrates extensive knowledge and understanding of operating systems concepts relevant to the question)
(Provides an insightful discussion of the reasons why multilevel dispatching schemes are important for process management)
(Provides an insightful comparison of the HOSTD dispatcher and other multilevel dispatching schemes used in at LEAST TWO different real operating systems)
(Outlines a range of shortcomings of the HOST dispatcher and possible improvements to address them with clear details justification)
(Refers to well-chosen examples to inform response)
(Presents a sustained, cohesive and logical response using appropriate information and ideas from a range of sources)

d. Discuss why such a multilevel dispatching scheme would be used, comparing
it with schemes used by `real` operating systems. Outline shortcomings in such a
scheme, suggesting possible improvements. Include the memory and resource
allocation schemes in your discussion
\fi

\end {document}

(Provides a thorough description of ALL prescribed memory allocation schemes - 6 marks)
(Provides an informed discussion of the advantages and disadvantages of these schemes)
For each: pros, cons, examples
a. Describe and discuss what memory allocation algorithms you could have used
and justify your final design choice
\fi

A number of memory allocation algorithms were available to choose from for this assignment, and each one will be detailed below - with a description of what they do, their advantages and disadvantages, and a worked example that will be consistent across each algorithm, using the specific version of the algorithm detailed in each subsection.\\

The algorithms that were allowed to be chosen from were First Fit, Worst Fit, Next Fit, Best Fit, and the Buddy System, all of which at least partially fall under the category of `Dynamic Partioning` (the latter being in a sense a `combination` of fixed and dynamic partioning), in contrast to `Fixed Partioning`. The former allows a variable number of partitions as well as their sizes, while the latter has both a fixed number of partitions as well as their sizes (although fixed sizes does not necessarily indicate fixed \textit{equal} sizes - unequal fixed sizes also still fit the category as well).\\

At this point, part of the narrowing down of algorithm choice has already been done for us due to the limitations of the specifications of the task - the exclusion of Fixed Partioning algorithms means that we automatically avoid some of their main disadvantages, including having a fixed (and hence) inflexible number of processes that can be running at any one time, inefficient use of partition space by smaller jobs, and internal fragmentation within partitions. Not only this, but virtually no real world systems still employ the use of fixed partioning, a clear indicator of its ineffiencies and pitfalls.\\
\iffalse reference the textbook here, page 314\fi

For our available class of algorithms, they share a number of properties which for the sake of brevity will be outlined in the introduction here rather than repeated in each subsection. Memory blocks managed through dynamic algorithms do not suffer from \textit{internal fragmentation}, but rather, \textit{external fragmentation}. Whereas internal fragmentation is a more inherent problem that cannot be solved without major modifications to an algorithm itself, external fragmentation is an issue that can be solved more simply (although not without a cost) - through the use of compaction. While internal fragmentation alludes to unused space within fragments, external fragmentation is the space between allocated partitions which are unused (and potentially can't have any pending jobs slot into it). Compaction solves this by periodically (based on a factor determined by the algorithm) by packing all the memory together back into a contiguous block, external fragmentation is at that moment in time eliminated - though this has issues of its own, as it requires both time and processing power, but more importantly, the now-modified locations of memory blocks can't invalidate references that previously existed which programs relied upon.\\

The following algorithms are named based on the way in which they allocate memory to incoming programs' requests.

\subsection{Best Fit}
The \textbf{Best Fit} memory allocation algorithm, as the name suggests, will find the smallest block available that is able to fit the request which it is given, starting from the beginning. This means that each search for an appropriate block is of $O(n)$ complexity, as an entire memory block must be traversed to determine the best-fitting block. This algorithm performs dismally, as always finding the best block means that the resulting \textit{size} of unused space (external fragmentation) is minimised - the wording of which can lead one to incorrectly believe is a positive thing, but in fact, rapidly creates a large number of miniscule, unusable blocks of memory, requiring compaction, which, as described above, is an expensive process which we want to minimise - while the Best Fit algorithm essentially maximises it instead. As a result, the best fit algorithm is in fact the worst performer of the pure dynamic allocation algorithms.\\

\textbf{However}, it is worth noting that despite its theoretical worst performance, this worst-case behaviour is often not observed, and when these cases do not surface simply due to the nature of the programs making requests to a block of memory, it doesn't necessarily perform worse than some of the other dynamic memory allocation algorithms. While some sources will find this point contentious, some (and possibly a majority) will actually claim that research has shown that Best Fit and First Fit aren't actually superior to each other in terms of wasted memory (though the latter holds) 

\subsection{Next Fit}
The \textbf{Next Fit} memory allocation algorithm will remember the last spot in which memory was last allocated (naturally, this will be the beginning if any memory is yet to be allocated), and serve requests by finding the next available block that is sufficiently large - taking $O(n)$ time. Under normal operation, this means that what usually happens is the largest block at the `end` of memory will be quickly fragmented as processes are finishing and their allocated memory freed, and new processes with different requirements take their place. This points to a relatively frequent need to compact memory, though less so than the theoretical worst case that the Best Fit algorithm requires.

\subsection{First Fit}
The \textbf{First Fit} memory allocation algorithm search from the beginning of a memory block to find the first block that is sufficiently large to satisfy the request being made. This traversal has a complexity of $O(n)$. Due to searching from the start every time, it takes longer than Next Fit to require breaking up blocks in the later areas of memory, which also means that theoretically, fragmentation will occur slower and hence compaction doesn't have to happen as frequently, reducing the overhead of the algorithm in this regard. However, though it holds the same complexity as Next Fit for each search, the \textit{actual} time to search is slightly higher as it has to check from the start each time, rather than simply the remaining memory left from the last allocated location. This means that during operation when not in the process of compacting, the First Fit algorithm is slower than the Next Fit algorithm. 

\subsection{Worst Fit}
The \textbf{Worst Fit} memory allocation algorithm will search from the beginning of a full memory block and find the block that is largest in difference to the request actually being made - hence it being named `worst fit`, and takes $O(n)$ time per search. Due to the requirement from searching the entire memory space each time a request is made, its speed during operation is the same speed as that of Best Fit.\footnote{D. Samanta. `Classic Data Structures` 2004. p. 76} However, its major difference to the Best Fit algorithm is that rather than trying to minimise `wasted` space, i.e. maximise small, fragmented blocks, it does the reverse - in place of the theoretical `small, unusable` blocks created by Best Fit, Worst Fit will instead have the largest block remaining possible, leaving them available for later processes that make larger requests for available memory.

\subsection{Buddy System}
The \textbf{Buddy System} memory allocation algorithm differs from the above in that it neither purely uses fixed or dynamic partioning schemes, but rather, a sort of combination of both. This is due to its aim to circumvent the inherent negative aspects of both - to avoid limitation of the number of processes and potential inefficient use of memory of fixed schemes, and the constant need for compaction and its computational cost of dynamic allocation schemes.\\

The buddy system assumes that the total size of available memory is $2^{U}$, where U is a positive integer. To obtain a block of memory to serve a program's request, the smallest block that is greater than or equal to the request size that is also $2^{k}$ where k is a positive integer is allocated. This block is retrieved by taking the full $2^{U}$ block, and spawning two children of size $2^{U-1}$, each of which spawn children of $2^{U-2}$, etc...until the correctly size block is found, at which point it is given to the program requesting it. It is apparent that this represents a binary tree - which is one of the data structures in which the Buddy System can be represented by. Hence, the complexity of allocation and freeing of memory is $log(n)$, a considerable increase in speed over the previous dynamic allocation methods detailed.\\

In its unadulterated form, the reader would be correct in observing one critical flaw with the buddy system - that it further exaggerates the problem of internal fragmentation, arguably beyond that of any reasonably designed fixed partitioning scheme - if we were to start with a 4096MB block of memory, and a program requests 2049MB, that would render the entire block in use, leaving 2047MB empty, yet unusable. This can be remedied through the use of slab allocation - a concept beyond the scope of the course.\\

\subsection{Algorithm Selected}
The algorithm that was selected for implementation in the assignment was the \textbf{Buddy System}. The reasons for this were both its superior speed when allocating, in comparison to the previous dynamic partitioning schemes, and eliminating the issues of external fragmentation, although it instead has issues of internal fragmentation - which can be fixed by other means. In the context of the assignment, however, this fact can be overlooked (or, more accurately, slab allocation does not need to be implemented) - and it is enough to identify that a solution exists for it. The other major advantage of choosing Buddy is that it is computationally cheaper to free memory - and seeing as the assignment specification says up to 1000 jobs can be in a particular instance, the constant freeing of jobs would impact performance (although not in a way that the user would notice given the nature of the application and the way jobs are `assigned`).\\

Upon further investigation, it can also be seen that many real-world systems even today use the Buddy System at least in part, in their memory allocation schemes. For example, the open source library \textit{jemalloc} uses the buddy system\footnote{people.freebsd.org/~jasone/jemalloc/bsdcan2006/jemalloc.pdf}, and is used in very well known projects today, such as the Firefox browser for managing fragmentation during its use on Windows platforms\footnote{https://github.com/jemalloc/jemalloc/wiki/History}, as well as the FreeBSD OS.

\iffalse (Justification of choice - identifies memory allocation algorithm used in code and clearly justifies choice in context of assignment - 2 marks) \fi

\pagebreak
\section{Queuing, Dispatching, Memory Allocation, and Resource Allocation Structures}

\iffalse
(Provides thorough description of ALL dispatcher structures that are relevant to host dispatcher implementation - 5 marks + 1 bonus)

b. Describe and discuss the structures used by the dispatcher for queuing,
dispatching and allocating memory and other resources.

structs, linked lists, [trg]
\fi

All the structures that were used in the implementation of the host dispatcher, including both the main program and the modules which it relied on, comprise of \textit{structs}, \textit{singly linked lists} which were used as queues in all instances where they appeared, \textit{arrays}, and \textit{binary trees}.

\subsection{Queueing and dispatching processes}
\textit{C structs} were used for a Process Control Block(\textit{Pcb}) which held the data for processes, containing their process ID, arguments (for the default `process` process that simply prints incremental `ticks` at a rate of 1/sec), arrival time, priority, remainingcputime, (psuedo) memory required in MB, a C struct containing resource requirements, status, a pointer to a C struct containing memory block information, and a pointer to the next \textit{Pcb}.\\

To actually queue and dispatch these processes (represented using Pcb's holding process info), singly lniked lists were used, as we only ever needed to take Pcb's from the front of queues, never in the middle of one or from the end. A Realtime Queue linked list was used to hold real time processes, an array of Feedback Queues were used to hold the feedback queues, with its indices offset by one compared to each job's actual priority (priority 1 jobs were in the feedback queue at index 0 in the array, priority 2 at index 1, priority 3 at index 2).\\

\subsection{Allocating memory}
As mentioned in \textit{Queueing and dispatching processes}, C structs were used to hold information pertaining to memory blocks (Memory Allocation Blocks, or Mabs) - i.e., their offset in main memory, the block's size, amount allocated (though this does not include memory allocated in any children blocks). Because the Buddy System was used for managing memory, a binary tree was the data structure of choice to maintain the available memory - and hence, each Mab also contains a pointer to its pointer Mab, as well as its left and right child Mabs, where they existed. Two binary trees were maintained - one for user jobs, and one for realtime jobs. 

\subsection{Allocating resources}
Once again, as mentioned in \textit{Queueing and dispatching processes}, C structs were used to hold information regarding required resources for a particular Pcb - i.e., the printers, scanners, modems, and cds it needed. While each Pcb contains a resource struct to keep track of the number of resources required, one more Resource struct existed in the main program to keep track of the available resources in the overall system at any one time.

\subsection{Appropriateness}
Due to C being a non-object oriented language, these were the best structures to use given the problem at hand. Queues are explicitly stated in the assignment spec, and hence queues were needed - implemented as singly linked lists. The only way to hold blocks of information together in C is using structs, hence their use for multiple parts of the program. The very nature of the Buddy System points towards it using a binary tree, which was why that was also used - and many sources verify this too, not the least of which includes the Art of Programming Volume 1.\footnote{D. Knuth. `The Art Of Programming: Volume 1 Fundamental Algorithms, Third Edition` 2013. p. 650 (ebook Edition)}\\

Some alternatives do exist to the structures used. On a more detailed level, while the implementation that this Design Document described used nested structs, all the information from each struct could have been instead all compressed into the one struct - in this way, memory would have been allocated marginally more contiguously, but this difference would have been so minor that it would be more advantageous for the sake of code clarity to separate information into appropriate C structs.\\

On a more fundamental level, the problem at hand would have lended itself nicely to the use of classes in C++ as well - something which could have been emulated in a bastardised form of C, by containing the functions relating to the purpose of each struct with the use of function pointers. However, this goes against common usage of C, as well as arguably violating using `straight` C, as the assignment required.

\pagebreak
\section{Program Structure}

\iffalse
(12 marks + 3 bonus)
(Provides comprehensive description of overall program structure, its modules, and major functions)
(Provides detailed and clear justification of program structure in context of assignment)

c. Describe and discuss the overall structure of your program, describing the
various modules and major functions (description of the function interfaces are
expected)
\fi

\subsection{Overall Program Structure}
The overall program structure of the Host Dispatcher utilises three modules - that of the Process Control Block, the Memory Allocation Block, and Resource block modules. This results in the code base in its entirety representing a modular, logical structure, accurately breaking down the problem into its constituents in terms of each individual module. By structuring the code around the inherent structure of the problem itself, it makes it easier to understand at first glance, as well as to modify when parts of the host dispatcher (or program in general) needs to be modified or extended. This may not be as apparent when looking at this assignment in isolation, but for argument's sake, if there were subsequent assignments where we received other students' code and were required to perform major extensions on it, well-implemented modularity would make it a much less daunting task. As a better (and more real-world relevant) example, perhaps a much larger program needs to incorporate this relatively minuscule project into it - and if it needed to use individual elements of it, the programmer could look directly to the appropriate module, rather than digging through a single, large file, which would likely be a monolithic, unorganised mess.\\

The main program contains several key elements:
\begin{itemize}
\item An \textbf{inputqueue}, which takes in all jobs from an input text file
\item A \textbf{userjobqueue}, which holds all user jobs from the inputqueue
\item An \textit{array of pointers} to \textbf{feedback queues}, containing jobs from the \textbf{userjobqueue} that have been supplied the sufficient amount of memory and resources they require
\item A \textbf{realtimebuffer}, which contains all real tmie jobs from the inputqueue
\item A \textbf{realtimequeue}, which contains jobs from the realtimebuffer that have been given enough memory to run
\end{itemize}

And without getting into any code or even too deep into specifics, the general flow of the program is as follows (note that the following list is not intended to act as instructions to follow to be able to implement the problem at hand to completion and perfection - but rather, to get a concise idea of how the problem was approached - hence, it is missing implementation specific details such as reading in files, specific error handling, actual flow logic, etc.). Steps $2$ and $3$ represent the Real Time Dispatcher module, while steps $4$ and $5$ represent the Feedback Dispatcher module.
\begin{enumerate}
\item Place all jobs from the inputqueue into the \textbf{userjobqueue} and \textbf{realtimebuffers} respectively
\item Allocate memory to jobs at the front of the realtimebuffer as long as it is available
\item Run all jobs in the \textbf{realtimequeue} from beginning to end until completion (hence skipping the following steps until such time is reached)
\item Allocate memory and resources to jobs in the \textbf{userjobqueue} as they become available, and enter it in the appropriate feedback queue (array) based on the specified priority
\item Run the process at the front of the highest priority feedback queue for one quanta, before placing it at the end of the next lower priority feedback queue
\end{enumerate}

The following subsections will detail each individual module and its major functions, and by doing so the way each one ties into the functionality of the program as a whole will become apparent. Note that trivial helper functions have been excluded, as they provide no useful insight into how the program functions, even to someone looking to utilise the program as a whole needing to understand roughly how everything operates and weaves together. Unlike the previous paragraph, however, considerable detail will be given with regards to each modules' functions, as they are the foundation of the main program.

\subsection{Process Control Block (pcb) Module}
\textbf{startPcb}\\
This function takes a Pcb pointer and checks the process' status - it will then either start an unstarted process using C's \textit{fork} and \textit{exec} functinos, or continue a paused one with a \textit{SIGCONT} signal- and this is done by checking the process ID (which is stored in a Pcb's struct) - a zero ID indicates a new process, while a non-zero one indicates it already exists but has simply been paused by a \textit{SIGTSTP} signal. The function's return value indicates whether it carried out its operation successfully - a vaild pointer indicates success, while a \textit{NULL} pointer means that the start/restart operation it attempted failed. This function gets called whenever a process is first being executed from a feedback queue, or being resume from a paused state. As such, there is no `resumePcb` function, as the reader may have observed.\\

\textbf{suspendPcb}\\
This function takes a Pcb pointer and sends the process it stores a \textit{SIGTSTP} signal, causing it to pause. The function will similarly return a NULL pointer if the signal fails. This function is called only on jobs in feedback queues, each time it has executed for one quanta.\\

\textbf{terminatePcb}\\
This function takes a Pcb pointer and sends the process it stores a \textit{SIGINT} signal, causing it to die. It is only called when a program's remainingcputime has reached zero.\\

\textbf{enqPcb}\\
This function takes a pointer to the head of a queue (specifically, a Pcb), and a Pcb pointer to the process block that is to be queued. The head of the existing queue will be stored in a temporary variable, and then modified to point at the back of the queue, allowing the last item in the queue to then be pointed at the target Pcb to be enqueued. The \textit{head} of this new queue is then returned, or in the case where a queue is yet to be created, simply the \textit{process} to be enqueued is returned instead - as it is the head of this newly created queue. This function is called whenever Pcbs need to be enqueued - whether it is into the \textbf{userjobqueue}, from this to the \textbf{realtimebuffer}/\textbf{userjobqueue}, to the \textbf{realtimequeue}/\textbf{feedbackqueues} respectively, or from one \textbf{feedbackqueue} to another.\\

\textbf{deqPcb}\\
This function takes a double pointer to the head of a queue (specifically, a Pcb), and takes the frontmost item off the queue by assigning a temporary pointer to the current head, and then pointing the head to the next item (in this case, the next of the temporary Pcb pointer). Hence, the new head after dequeuing will be returned if successful, or NULL returned if unsuccessful, when a queue is already empty. With the exception of when the inputqueue is being loaded with data from an external text file, deqPcb will always be called in pairs with enqPcb - a Pcb is only enqueued onto one queue after being dequeued from another.\\

\textbf{checkRsrcs}\\
This function takes a pointer to a Pcb and checks whether it requests valid resources. The role of this function is twofold, as it checks for both realtime and user jobs. It returns true for the former only if no resources are requested, whereas it returns true for the latter so long as the requested amount doesn't exceed the system's total available resources (not simply the amount available at the time - that allocation is handled in the main program). This function is called whenever a process is trying to obtain resources in order to be queued to run.
\\

\textbf{checkMem}\\
This function takes a pointer to a Pcb and checks whether it requests valid memory. The role of this function is similarly twofold, as it checks for both realtime and user jobs. It returns true for the former when memory requested doesn't exceed 64mb, whereas it returns true for the latter so long as it doesn't return more than half the available size. This seemingly abnormal restriction is because of the naive implementation of the Buddy System, and reserved memory for realtime jobs - the largest available block for a system with 1024MB memory using this implementation of the Buddy System is only 512MB. This function is called whenever processes are trying to reserve a block of memory in order to be queued to run.\\

At this point, multiple justifications are in order regarding some of the design choices made above. 
\begin{itemize}
\item There is a generalised check function despite realtime jobs and user jobs requiring different checks - the programmer using the interface should not have to worry about this, as it involves looking at information in the Pcb (determining its type based on priority), or checking the surrounding logic in the code - rather, they should be able to call a check on any Pcb and trust that it will determine validity of its memory/resource requests regardless of type
\item The check functions do not simultaneously allocate memory/resources if they are available. This is necessary, as both resources and memory must be available at any point in time before they are allocated - if each pre-allocates them within their isolated functions, we will end up with cases where one allocates memory/resources and the other doesn't because it wasn't able to - resulting in either a bug, or extra code to do otherwise unnecessary freeing if both conditions are satisfied at once
\item This point must may have occurred to the reader as soon as they happened upon the check functions - why are resource and memory checks not in the resource and memory modules? This is justified and even preferred, because first of all - they should take pointers to Pcbs and not Rsrc/Mab blocks specifically, as the programmer using these interfaces should not have to worry about such specifics. Secondly, given the previous assertion, we can't place the functions in the memory/resource modules as circular dependencies would arise as a result
\end {itemize}

\subsection{Memory Allocation Block (mab) Module}
\textbf{blockSizeNeeded}\\
This function takes an integer, and finds the smallest power of two that is greater than it, and returns the value. This function is used throughout the Memory Allocation Block module to determine sizes needed via the Buddy System.\\

\textbf{memChk}\\
This function takes a pointer to a Mab - the root node of the Buddy tree, and an integer, the size being requested, and checks if an available block of such a size exists. It does this by recursing into the left then the right trees, seeing if there is a valid block that can be allocated. It will continue to spawn child nodes at each node, each containing half the size of its parent, until one is found, in which case the pointer to this valid block is returned. If no valid block can be found in the tree, the function returns NULL. It is called when a process wants to first check if there is available memory before allocating it.\\

\textbf{memAlloc}\\
This function takes a pointer to a Mab and an integer whose size is that needed by a process, and essentially makes a call to memChk and allocates it if a valid block is returned. A pointer to the Mab containing the allocated block is returned if successful, or NULL otherwise. The function is called once resource/memory requirements have been confirmed to be satisfied, and memory can be safely allocated.\\

\textbf{memFree}\\
This function takes a pointer to a Mab - the one which is to be freed. It marks its allocated amount as zero, then checks that it doesn't have children (which would indicate that one of its child or indirect child nodes has allocated memory, due to the way this module is written), and also that its buddy doesn't have children. If this is the case, both the block to be freed and its buddy will be freed - and this process is repeated recursively until all free blocks have been merged back into one larger block. Due to the method calls made from \textbf{memFree}, it is guarded from errors, and the function simply returns a NULL pointer once it has freed all it can. This is called in conjunction with whenever processes are terminated.\\

\textbf{createUserMem}\\
This function takes no arguments, and creates a block of memory equal in size to all the available memory, and returns a pointer to the Mab block - which acts as the root node of the Buddy System memory tree.\\

\textbf{createRTMem}\\
This function takes no arguments, and creates a block of memory equal to the max size available for real time jobs (in our case, 64MB), and returns a pointer to the Mab block - the root node of the Buddy System memory tree.\\

Once again, some justifications need to be made regarding the design choices of this module:
\begin{itemize}
\item It may appear that the Buddy System memory tree is being damaged when checks are performed, which may render later use of it invalid - however, due to how it is implemented here, this is inevitable - if a check is performed but an allocation not made, the changes to the tree must be rolled back. While this may seem expensive, these operations have a complexity of $O(logn)$ - two operations of this complexity are still far lower in complexity than the consistently $O(n)$ operations of previously mentioned dynamic partionining schemes
\item While the previous module combined user and realtime checks in a single function, operations exclusive each have been separated here - this is because they require different properties on creation, and a generic interface would have resulted in the programmer using the interface to explicitly specify the size of the root node - an implementation detail which they should be shielded from
\item The memFree function appears that it does not need to return a value at all. However, there are many possible implementations of the Buddy System, and while this one did not need to return a meaningful value in that function others may need to - and a future change reflecting this would require modifying function headers and modules had the function been of \textit{void} type, potentially causing breaking changes in the main program as a result (though not in this case, as the assignment has been completed and is not being passed onto a future party to build on)
\end{itemize}

\subsection{Resource Block (rsrc) Module}
It is worth mentioning that the functions in this module are trivial in nature and complexity, and the reader may wish to simply skim over it or even skip it if they so wish.

\textbf{rsrcChk}\\
This function takes a pointer to a Rsrc, the system-wide availibility master block, and a Rsrc, containing the resources required by a particular process. It checks if the resources requested are avaiable, and returns the pointer to the master block if they are, or NULL if they are not. This function is called when processes are checking if enough resources exist for them to be executed.\\

\textbf{rsrcAlloc}\\
This function takes a pointer to a Rsrc, the master block, and a Rsrc, containing the resources required by a particular process. The resources needed are substracted from the master block. A pointer to the master block is returned on success. This function is called when there exist sufficient resources and memory and a process is ready to be executed.\\

\textbf{rsrcFree}\\
This function takes a pointer to a Rsrc, the master block, and a Rsrc, containing the resources used by a particular process. The resources used are then added back to the master block. A pointer to the system-wide available resource block is returned on success. This function is called when a process finishes and is ready to release their resources.\\

\textbf{createRsrcs}\\
This function takes no arguments, and creates the system-wide available block of resources, returning a pointer to the Rsrc block. A pointer to this block is returned on completion. This function is called once at the beginning of the program to initialised the system's available resources.\\

Once again some justifications regarding design choices in this module, some of which mirror the above ones:
\begin{itemize}
\item The allocation and freeing functions don't seem as they they could possibly fail, making the return values pointless - but this is only the case because of the diluted capability of the program. In reality, allocating resources is not as simple as incrementing and decrementing counters, and I/O operations themselves could have issues - and return values in those instances would have value, which is what the module here is aiming to represent
\end{itemize}

\pagebreak
\section{Discussion - Use of Multilevel Dispatching Scheme}

The feedback queue component of the task utilises a multi-level dispatching scheme. These are important for process management because they allow the prioritising of different types of processes requiring different resources (e.g. CPU bound or I/O bound) to be adequately intricate such that throughput and performance are maximised. In recent times (though not technologically speaking), Windows NT was one particular system that utilised such a scheme. This ties in with one key weakness of the algorithm being used in the task at hand - equal quanta are assigned to CPU-bound tasks are they are to those requiring I/O operations - but because by their nature, I/O tasks are more expensive and time consuming, it is possible that quite often they only ever use part of their quanta, sacrificing the rest without ever getting it back, which inadvertently favours CPU-bound tasks over I/O-bound ones. One possible solution to this is the one which Windows NT employed - by adjusting priorities (naturally, priorities in a real operating system are likely to be more fine-grained then the coarse 3-level one used in our dispatcher) based on whether a process favoured I/O or CPU intensive tasks, and gave higher priority to the former.\footnote{"A Tale of Two Schedulers Windows NT and Windows CE", http://web.archive.org/web/20120722015555/http://sriramk.com/schedulers.html}\\

This is not the only solution to the problem, however - rather than simply assigning higher priority to I/O bound processes to try and balance the scales in terms of time given to processes, which would involve estimating how long certain I/O operations would take relative to their CPU counterparts before adjusting priorities accordingly which would provide them even time, a rather complex calculation, there exists a simpler alternative. This would be to actually take the time spent by an I/O-bound process after the fact, and then take the leftover time, i.e. subtracting the time spent by the process from the quanta that was actually available to it, and storing that value for that particular process. The I/O-bound process, if it hadn't used up its quanta, is then placed on a special queue for I/O-bound processes which are allowed to execute for a period equal to the sum of time equal to its leftover each time a quanta wasn't fully utilised. This scheme is also utilised in real-world systems - one particular example being Solaris.\footnote{pages.cs.wisc.edu/.../solaris-notes.pdf}.\\

Both Window NT and Solaris present viable solutions to problems which are very much inherent in our implementation of a multilevel dispatcheing scheme - that being that we don't intentionally discriminate between I/O-bound and CPU-bound processes, and as such the natural CPU-bound skew will arise, causing potential starvation for I/O-bound processes. This only accounts for one flaw in our scheme - an even larger, more critical problem, arguably, is the high possibility of starvation in our naive implementation. In our three-level dispatcher queue, once a process has reached priority level three, it stays there with no way to increase its priority - this means that if a constant stream of processes request resources that are available (likely to be low resource-requirement with low total execution times), processes that drop to the lowest priority will never get serviced. While the previous issue addressed starvation of I/O-bound processes specifically, we have now identified that our model also has the issue of starvation of processes in general.\\

The most obvious solution to this issue is the use of aging - which observes the age of a process within its particular level in a multilevel feedback queue, and based on metrics defined by the designer of the system, can raise the priority of a process depending on its age within a particular queue, to prevent starvation. Depending on how old it is, it could be raised by $k$ priorities, proportional to its age. While the current implementation doesn't lend itself perfectly to this solution, the fix to do so is an easy one. Because the \textbf{realtimequeue} is currently separate from the array of feedback queues, and the structure of the main program unquestioningly gives absolute priority to jobs in the \textbf{realtimequeue}, raising priorities of user jobs won't make a difference if there exists a constant stream of realtime jobs. Combining the \textbf{realtimequeue}s and \textbf{feedbackqueue}s into a single array could present a viable remedy - first by having realtime jobs at index $0$in the array, and processes with current priority $k$ being in queues of $1...n$ where $1 <= k <= n$. For queues at index $1$ onwards, an age constant could then be used for each process in the queue where once it hits a certain number, will increase its priority by one and hence enter the queue at priority $k-1$ where $k$ is the queue's index which the process waited in before its age reached the threshold. An example of this in practice would be that if priorities ranged from 0 to 127\footnote{Silbershatz, Galvin, Gagne Operating System Principles, 7th ed., p.163}, we could raise the priority of a waiting process by one every second (the amount suggested in the previously referenced book was 15 minutes, an obtusely unrealistic amount for an average user system) - in which case, even a priority 127 task would be executed in a little over two minutes.\\

The Buddy System that was used in this implementation is rather flawed, as it is very suceptible to external fragmentation - if a process requires $2^k+1$ MB of memory, it will take up a block of size $2^{k+1}$ - resulting in $2^k-1$ MB of wasted space. However, as previously mentioned, this is an issue that can be largely alleviated using slab allocation, and as a result returns it to its superior status over pure dynamic partioning schemes. A number of modern systems (or modern systems in their previous versions, at least) use a combination of Buddy System and SLAB allocation in their systems due to its strong performance, including FreeBSD, NetBSD, Solaris Linux, and the Perl 5 Compiler. Upon further reflection, it would appear that some of the implementation details in this program's Buddy System, is also non-optimal, such as finding buddies through the parent rather than performing and addresses on a block's address to find it, etc. A more efficient implementation could have been followed in Donald Knuth's Art of Programming Volume 1, as mentioned earlier in the report.

\iffalse
(20 marks + 5 bonus)
(Demonstrates extensive knowledge and understanding of operating systems concepts relevant to the question)
(Provides an insightful discussion of the reasons why multilevel dispatching schemes are important for process management)
(Provides an insightful comparison of the HOSTD dispatcher and other multilevel dispatching schemes used in at LEAST TWO different real operating systems)
(Outlines a range of shortcomings of the HOST dispatcher and possible improvements to address them with clear details justification)
(Refers to well-chosen examples to inform response)
(Presents a sustained, cohesive and logical response using appropriate information and ideas from a range of sources)

d. Discuss why such a multilevel dispatching scheme would be used, comparing
it with schemes used by `real` operating systems. Outline shortcomings in such a
scheme, suggesting possible improvements. Include the memory and resource
allocation schemes in your discussion
\fi

\end {document}
